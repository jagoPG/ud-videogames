\chapter{Games}
Still has not passed enough time to implement emph{Vulkan} in a huge variety of games. But there are some companies that
have implement the \gls{api} for develop their engines.

The first company to launch a game supporting this \gls{api} has been \emph{Oxide Games} with the game \emph{Ashes of
Singularity}, a strategy on real time game. It has been the first game to support \emph{DirectX 12} too. It is strange
that they have released the game under Windows, having the chance of supporting other platforms.

\emph{Doom}, is one of the first emph{AAA} games that implemented \emph{Vulkan} a few months after it was put on sale.
Normally, \emph{ID Software}, the company that has developed this saga, has typically developed their games with the
\emph{OpenGL} \gls{api}. In this case, they have develop the game using their engine \emph{ID Tech 6} - based on
\emph{OpenGL} - and then they have added support to \emph{Vulkan} with excellent results.

\emph{Valve} has implemented \emph{OpenGL} in their engine \emph{Source 2}. So that games as \emph{Dota 2} have been
the first titles to offer multiplatform support. \emph{Ark: Survival Evolver} is other game offering multiplatform
support with an excellent quality.

Other videogame examples that implement \emph{Vulkan}, in desktop as much in mobile devices, are \emph{Vainglory}
(Figure \ref{fig:vainglory}), \emph{Rust}, \emph{Need for Speed: No Limits}...

\begin{figure}[t]
	\begin{center}
		\includegraphics[scale=0.3]{galaxy_s7_edge-vainglory}
		\caption{\emph{Vainglory} running on a \emph{Samsung Galaxy S7 Edge}}
		\label{fig:vainglory}
	\end{center}
\end{figure}
