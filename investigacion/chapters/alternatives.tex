\chapter{Alternatives}
Aunque ya he mencionado varias alternativas a Vulkan, voy a mencionar algunos de los kit de desarrollo que se emplean
en diferentes sistemas y sus particularidades.

\section{Metal}
Es una \gls{api} que provee un acceso de alto rendimiento y bajo nivela la \gls{gpu}, permitiendo maximizar los
gráficos y el potencial de computación de las aplicaciones en iOS, macOS y tvOS. Posee una \gls{api} simplificada,
shaders precompilados, y soporte multihilos. Se dirige sobre todo a los desarrolladores que quieren llevar al límite
las tarjetas gráficas integradas en los dispositivos móviles de Apple.

Fue liberada en el año 2015 como una alternativa a OpenGL, buscando permitir un acceso más directo al hardware del
dispositivo - un mismo objectivo que Vulkan. Con ello se quiere reducir el consumo de recursos y los tiempos de
acceso al \emph{metal}. Entre algunas de las característica que se añaden, es una mayor sincronización entre la
\gls{gpu} y la \gls{cpu} para que entre los dos dispositivos tengan que intercambiarse menor información. Esto se basa
en la premisa de que cada vez que es necesario que la tarjeta gráfica dibuje algo, la \gls{cpu} tiene que mandarlo
explicitamente. Mediante la compartición de recursos, la gráfica puede acceder directamente a los datos que sean
necesarios sin requerir que el procesador se los tenga que mandar.

\section{OpenGL}

\section{DirectX}
