\chapter{Alternatives}
Aunque ya he mencionado varias alternativas a Vulkan, voy a mencionar algunos de los kit de desarrollo que se emplean
en diferentes sistemas y sus particularidades.

\section{Metal}
Es una \gls{api} que provee un acceso de alto rendimiento y bajo nivela la \gls{gpu}, permitiendo maximizar los
gráficos y el potencial de computación de las aplicaciones en iOS, macOS y tvOS. Posee una \gls{api} simplificada,
shaders precompilados, y soporte multihilos. Se dirige sobre todo a los desarrolladores que quieren llevar al límite
las tarjetas gráficas integradas en los dispositivos móviles de Apple.

Fue liberada en el año 2015 como una alternativa a OpenGL, buscando permitir un acceso más directo al hardware del
dispositivo - un mismo objectivo que Vulkan. Con ello se quiere reducir el consumo de recursos y los tiempos de
acceso al \emph{metal}. Entre algunas de las característica que se añaden, es una mayor sincronización entre la
\gls{gpu} y la \gls{cpu} para que entre los dos dispositivos tengan que intercambiarse menor información. Esto se basa
en la premisa de que cada vez que es necesario que la tarjeta gráfica dibuje algo, la \gls{cpu} tiene que mandarlo
explicitamente. Mediante la compartición de recursos, la gráfica puede acceder directamente a los datos que sean
necesarios sin requerir que el procesador se los tenga que mandar.

\section{OpenGL}
Especificación de estándar que define una \gls{api} multiplataforma y multilenguaje. El lanzamiento inicial fue
realizado en 1992 como una solución a los problemas gráficos que había desde los años 80, cuando se tenían que
subcontratar programadores para escribir drivers específicos para cada tipo de hardware.

Durante mediados de los años noventa, se intentó unificar las tecnologías Direct3D de Microsoft y OpenGL de manera
que las librerías gráficas creasen un nuevo estándar para la industria. Por falta de financiación y de apoyos este
proyecto fracasó a los pocos años de ponerse en marcha.

Su uso es libre y es un buen comienzo para los programadores que quieren iniciarse en el mundo de las librerías
gráficas. Actualmente no tiene una gran relevancia en los entornos de escritorio, donde pocas compañías usan este
estándar como primera opción para desarrollar. Es muy usado para la creación de motores gráficos, simuladores y
aplicaciones dentro de los dispostivos móviles.

\section{DirectX}
Es una colección de \gls{api} desarrolladas para facilitar las tareas relacionadas con multimedia, en particular la
programación de juegos y vídeos. Consta de diferentes \gls{api} para dibujar en 2D, 3D, procesamiento de dispositivos
de entrada, comunicaciones por red, reproducción y grabación de audio...

Tan solo está disponible para plataformas de Microsoft, como Windows o Xbox. Aunque se está desarrollando una
implementación de código abierto para sistemas Unix, este proyecto se llama WineHQ.

El lanzamiento inicial se realizó a finales de 1995. Antes de eso los juegos eran ejecutados directamente bajo DOS,
donde no se empleaba ninguna \gls{api} específica de desarrollo multimedia. Durante estos años el mercado demandaba
cada vez más aplicaciones multimedia, y no existía una solución nativa para este tipo de necesidades. La primera
versión de DirectX se integró en Windows 95, en sustitución de \gls{dci} y WinG \gls{api}s para Windows 3.1.

Direct3D está enforcado prácticamente a los juegos, durante años a ido ampliando con nuevas características y
optimizaciones esta especificación. Cuando pusieron a la venta las consolas Xbox realizaron una \gls{api} específica
para consolas. De la misma forma, DirectX está presente en su sistema móvil Windows phone. Microsoft aún da soporte a
OpenGL dentro de sus sistemas de escritorio, esto se debe a que aplicaciones como CAD lo emplean. Para aprovechar
las características de OpenGL, Microsoft ha creado un soporte intermedio entre DirectX y OpenGL para aprovechar lo mejor
de ambos mundos, ofreciendo a los desarrolladores diferentes opciones a la hora de implementar sus soluciones.
