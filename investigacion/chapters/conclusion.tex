\chapter{Conclusions}
\emph{Vulkan} is quite promising \gls{api} which has two strong points: it is supported by some of the most relevant
gaming companies and has an emphasis on the community to take part in the develop of libraries and improvements. This
makes the technology available to anyone wanting to put its bit. Furthermore speeds up the development of applications
due to the wheel is not reinvented in each new project.

The implementation of \emph{Vulkan} is called \emph{LunarG} and has all documentation and \gls{sdk} on its website.
Remember that the development tools are only available in \emph{Windows} and \emph{GNU\/Linux} and there is no support
for \emph{iOS} and \emph{macOS} systems.

The throughput of \emph{Vulkan} is a bit lower than \emph{OpenGL}, but the stability of the frame rate is better.
Companies have to improve their drivers due to this technology is new and requires time to optimize the graphical
libraries. Despite these facts, many companies have make their games compatible with this \gls{api}, which is a
remarkable point.

I expect promising future feature to be released soon. Meanwhile people's hardware has to be updated to be compatible
with \emph{Vulkan}.
