\chapter{History}
Khronos group is an industrial consortium dedicated to the creation of \gls{api}s with open standard and free. This
standards should allow the creation and spread of multimedia content on multiple devices. Some of their most relevant
projects are: OpenGL, a multiplatform \gls{api} oriented to graphics; OpenCL, an \gls{api} for multiplatform
computation; OpenSL, an optimized \gls{api} for 3D and MIDI reproduction on integrated devices.

In 2014, this group get together for a kickoff meeting with the Valve Corporation, where the creation of an API for the
next generation of graphic was proposed. In the SIGGRAPH conference that toke place that year - where people and
companies related with the computational graphics gathers - professionals of the multimedia sector wishing to
contribute to the project were summoned.

For creating this \gls{api} they were based Mantle, a specification created by AMD. The source code was relinquised
to Khronos with the intention of generating a new low level standard similar to OpenGL.

The following year, Khronos presented Vulkan in the \emph{Game Developers Conference 2015}. Until that moment the
\gls{api} was known as \emph{glNext}. One company bounded to Valve, called \emph{LunarG}, demonstrated how they
were able to develop a driver for GNU\/Linux using the graphical card Intel Graphics HD 4000 - despite that until that
date Mesa drivers were not compatible with OpenGL 4.0.

It won't take long until have acceptance. In August 2015, Google announced that Vulkan would be supported for the
incoming Android \gls{os} versions. From the version 24 of android this \gls{api} can be used. Beginning 2016
the first release version of Vulkan was published joined together to the specification.

Unity Technologies announced that from the version 5.6 of Unity, Vulkan would be supported.

In theory, Vulkan could be used for be employed with hardware of parallel computing, for controlling billions of
\gls{gpu} cores, in little wearable devices, 3D printers, vehicles, \gls{vr} and basically anything kit out with
a \gls{gpu}.

Nowadays, the development kit is available for Windows and GNU\/Linux systems. There is no plan of supporting MacOS but,
as is explained in section \ref{ch:features}, the community is developing a compatible \gls{api}.
