\chapter{History}
El grupo Khronos es un consorcio industrial dedicado a la creación de \gls{api}s con estándares abiertos y de uso libre
que permite la creación y reproducción de contenido multimedia en múltiples dispositivos. Algunos de sus trabajos más
destacados se encuentra: OpenGL, una \gls{api} multiplataforma para gráficos; OpenCL, una \gls{api} de computación
multiplataforma; OpenSL, una \gls{api} optimizada para dispositivos integrados de reproducción de audio 3D y MIDI.

Este grupo realizón en 2014 una reunión kickoff con la corporación Valve, donde se incubó un proyecto con el objetivo
de crear una \gls{api} para la próxima generación de gráficos. En la conferencia SIGGRAPH que se celebró ese año, donde
se juntan personas y empresas relacionadas con los gráficos computacionales, se convocó a profesionales que estuviesen
dispuestos a tomar parte en el proyecto.

Para crear esta \gls{api} se basaron en Mantle, una especificación ya creda por AMD, cuyo código fuente fue cedido a
Khronos con la intención de generar un estándar a bajo nivel abierto similar a OpenGL.

Un año después, Khronos presentó Vulkan en la \emph{Game Developers Conference 2015}, que era conocido hasta entonces
como glNext. Una empresa ligada a Valve llamada \emph{LunarG} demostró cómo habían desarrollado un driver para GNU/Linux
para la tarjeta gráfica Intel Graphics HD 4000, a pesar de que en esas fechas los drivers de Mesa no eran compatibles
con la versión OpenGL 4.0 hasta ese año.

No tardó mucho tiempo en tener una acogida, en agosto de 2015 Google anunciaba de que Vulkan iba a ser soportada por
futuras versiones de Android; a partir de la versión Android 24 puede hacerse uso de esta \gls{api}. A principios de
2016 se anunció la salida de la primera versión de Vulkan y su especificación.

Unity Technologies anunciaron que a partir de la versión 5.6 de Unity se soportaría Vulkan.

Teóricamente, Vulkan podría ser usado para usarse en hardware de computación paralela, para controlar billones de
núcleos de GPUs, en pequeños dispositivos wearables, en impresoras 3D, coches, \gls{vr} y básicamente en cualquier
cosa que tenga una GPU.

Actualmente el kit de desarrollo está disponible tanto para sistemas Windows como para GNU/Linux. De momento no existe
soporte por parte de MacOS aunque, como se explica mas adelante, hay planes por parte de la comunidad de crear una
\gls{api} (tal y como se hizo con OpenGL ES).
