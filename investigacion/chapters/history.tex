\chapter{History}
\emph{Khronos} group is an industrial consortium dedicated to the creation of \gls{api}s with open standard and free.
This standards should allow the creation and spread of multimedia content on multiple devices. Some of their most
relevant projects are: \emph{OpenGL}, a multiplatform \gls{api} oriented to graphics; \emph{OpenCL}, an \gls{api} for
multiplatform computation; \emph{OpenSL}, an optimized \gls{api} for 3D and \emph{MIDI} reproduction on integrated
devices.

In 2014, this group get together for a kickoff meeting with the \emph{Valve Corporation}, where the creation of an
\gls{api} for the next generation of graphic was proposed. In the \emph{SIGGRAPH} conference that toke place that
year - where people and companies related with the computational graphics gathers - professionals of the multimedia
sector wishing to contribute to the project were summoned.

For creating this \gls{api} they were based \emph{Mantle}, a specification created by \emph{AMD}. The source code was
relinquised to \emph{Khronos} with the intention of generating a new low level standard similar to \emph{OpenGL}.

The following year, \emph{Khronos} presented \emph{Vulkan} in the \emph{Game Developers Conference 2015}. Until that
moment the \gls{api} was known as \emph{glNext}. One company bounded to \emph{Valve}, called \emph{LunarG},
demonstrated how they were able to develop a driver for \emph{GNU\/Linux} using the graphical card
\emph{Intel Graphics HD 4000} - despite that until that date \emph{Mesa} drivers were not compatible with
\emph{OpenGL 4.0}.

It won't take long until have acceptance. In August 2015, \emph{Google} announced that \emph{Vulkan} would be
supported for the incoming \emph{Android} \gls{os} versions. From the version 24 of \emph{Android} this \gls{api} can
be used. Beginning 2016 the first release version of \emph{Vulkan} was published joined together to the specification.

\emph{Unity Technologies} announced that from the version 5.6 of \emph{Unity}, \emph{Vulkan} would be supported.

In theory, \emph{Vulkan} could be used for be employed with hardware of parallel computing, for controlling billions of
\gls{gpu} cores, in little wearable devices, 3D printers, vehicles, \gls{vr} and basically anything kit out with
a \gls{gpu}.

Nowadays, the development kit is available for \emph{Windows} and \emph{GNU\/Linux} systems. There is no plan of
supporting \emph{MacOS} but, as is explained in section \ref{ch:features}, the community is developing a compatible
\gls{api}.
