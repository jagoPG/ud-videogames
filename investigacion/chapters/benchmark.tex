\chapter{Benchmark}
Dado que Vulkan tiende a ser el futuro reemplazo de OpenGL es necesario comparar si Vulkan ofrece una respuesta, como
mínimo, equivalente a OpenGL. Para ello he realizado un benchmarking en dos juegos diferentes, una vez con cada
tecnología. He ejecutado las mimas escenas para evitar que una de las \gls{api} tengan que soportar menor carga
que la otra.

Las especificaciones del ordenador con el que he hecho las pruebas son las siguientes: procesador Intel Core i7-4790,
equipado con 16,0 GB DD3 de RAM a 1600 Mhz. y una tarjeta gráfica AMD Radeon R9 380 Series. He actualizado los drivers
de AMD a la última versión para disfrutar de las últimas mejoras que se han realizado.

Se ha empleado el software \emph{PresentMon64} para realizar las mediciones, el cual está libremente disponible en
el repositorio de GitHub \href{https://github.com/GameTechDev/PresentMon}{GameTechDev/PresentMon}. Con esta aplicación
se puede medir el tiempo que tarda en pintar la pantalla, tiempo que tarda en renderizar cada frame, tiempo de
sicronización...

\begin{table*}[t]
  \centering
  \begin{tabular}{p{0.15\linewidth}p{0.15\linewidth}p{0.15\linewidth}p{0.15\linewidth}p{0.15\linewidth}p{0.15\linewidth}}
    \toprule
    Juego & \gls{api}    & Media & Máximo & Mínimo & Desviación típica \\
    \midrule
    Doom 2016   & OpenGL  & 149 & 184 & 7   & 11.169 \\
    Doom 2016   & Vulkan  & 85  & 194 & 36  & 9.786 \\
    Dota 2      & OpenGL  & 137 & 200 & 6   & 33.17 \\
    Dota 2      & Vulkan  & 122 & 200 & 9   & 30.719 \\
    \bottomrule
  \end{tabular}
  \caption{Resultados del benchmarking. Se miden \gls{fps}.}
  \label{tab:benchmarking_results}
\end{table*}

Durante las ejecuciones de las pruebas no se ha notado ningún tipo de problemas a la hora de jugar, es decir la
experiencia de juego ha sido buena. A partir de los datos recogidos, se ha realizado la Tabla \ref{tab:benchmarkin_results}
donde se puede observar que de media OpenGL logra tener una tasa de frames superior a Vulkan. Sin embargo, Vulkan logra
mantener una mayor estabilidad de los frames. Ambas tecnologías tienen una media de frames superior a la que actualmente
el mercado demanda - que son unos 60 FPS jugando a resolución 1920x1080.

Hay que destacar que aún los drivers de Vulkan no tienen el recorrido de OpenGL, por lo que seguramente en el futuro
los resultados sean mejores. Como conclusión de esta prueba creo que actualmente es una alternativa totalmente
viable a OpenGL, al menos desde el punto de vista multimedia.
