\chapter{Benchmark}
Dado que Vulkan tiende a ser el futuro reemplazo de OpenGL es necesario comparar si Vulkan ofrece una respuesta, como
mínimo, equivalente a OpenGL. Para ello he realizado un benchmarking en dos juegos diferentes, una vez con cada
tecnología. He ejecutado las mimas escenas para evitar que una de las \gls{api} tengan que soportar menor carga
que la otra.

Las especificaciones del ordenador con el que he hecho las pruebas son las siguientes: procesador Intel Core i7-4790,
equipado con 16,0 GB DD3 de RAM a 1600 Mhz. y una tarjeta gráfica AMD Radeon R9 380 Series. He actualizado los drivers
de AMD a la última versión para disfrutar de las últimas mejoras que se han realizado.

Se ha empleado el software \emph{PresentMon64} para realizar las mediciones, el cual está libremente disponible en
el repositorio de GitHub \href{https://github.com/GameTechDev/PresentMon}{GameTechDev/PresentMon}. Con esta aplicación
se puede medir el tiempo que tarda en pintar la pantalla, tiempo que tarda en renderizar cada frame, tiempo de
sicronización...

\begin{table*}[t]
  \centering
  \begin{tabular}{p{0.15\linewidth}p{0.15\linewidth}p{0.15\linewidth}p{0.15\linewidth}p{0.15\linewidth}p{0.15\linewidth}}
    \toprule
    Juego & \gls{api}    & Media & Máximo & Mínimo & Desviación típica \\
    \midrule
    Doom 2016   & OpenGL  & 149 & 184 & 7   & 11.169 \\
    Doom 2016   & Vulkan  & 85  & 194 & 36  & 9.786 \\
    Dota 2      & OpenGL  & 132 & 145 & 85  & 10.84 \\
    Dota 2      & Vulkan  & 129 & 200 & 5   & 34.073 \\
    \bottomrule
  \end{tabular}
  \caption{Resultados del benchmarking. Se miden \gls{fps}.}
  \label{tab:benchmarking_results}
\end{table*}
