\chapter{Benchmark}
As long as Vulkan approaches to be an OpenGL replacement, it is required to compare if Vulkan is able to offer
a throughput greater than OpenGL - or at least similar. For this reason I have performed a benchmark with two different
games. With each game I have played twice a round, one time with OpenGL and another time with Vulkan. I have played the
same scenes with the two implementations of the technologies for avoid having less computing load with one of the
technologies. The game screen resolution has been set up 1920x1080, with the highest graphic quality.

The specifications of the computer I have done the tests with are the following: Intel Core i7-4790 processor, 16 GB
of DDR3 RAM running at 1,600 MHz. and an AMD Radeon R9 380 series graphic card. I have updated the AMD drivers to the
latest version for having the latest improvements.

I have used the software \emph{PresentMon64} for getting the measurements, which is free for use in the GitHub
repository \href{https://github.com/GameTechDev/PresentMon}{GameTechDev/PresentMon} at GitHub. This application can
measure the times that takes to draw the frame, time that takes to render the frame, synchronization time...

\begin{table*}[t]
  \centering
  \begin{tabular}{p{0.15\linewidth}p{0.15\linewidth}p{0.15\linewidth}p{0.15\linewidth}p{0.15\linewidth}p{0.15\linewidth}}
    \toprule
    Juego & \gls{api}    & Average & Maximum & Minimum & Standard Deviation \\
    \midrule
    Doom 2016   & OpenGL  & 149 & 184 & 7   & 11.169 \\
    Doom 2016   & Vulkan  & 85  & 194 & 36  & 9.786 \\
    Dota 2      & OpenGL  & 137 & 200 & 6   & 33.17 \\
    Dota 2      & Vulkan  & 122 & 200 & 9   & 30.719 \\
    \bottomrule
  \end{tabular}
  \caption{Bencharmk Results. Measured as \gls{fps}.}
  \label{tab:benchmark_results}
\end{table*}

During the test I have not notice any problem when playing - that is the game experience has been excellent. Based
on the collected data the Table \ref{tab:benchmark_results} has been constructed. As you can see, the average of
OpenGL has been greater that Vulkan average. However, Vulkan has a greater frame stability as the standard deviation
is lower than the OpenGL frame rate. Both technologies have a greater frame average than the frame rate the players
currently demand - about 60 \gls{fps} playing with a resolution of 1920x1080.

It has to be taken into consideration that Vulkan drivers have not the same maturity level as OpenGL. In the future
it is for sure that the Vulkan results will be improved. As conclusion of this test, I think that today Vulkan could
be a good alternative to OpenGL, at least from the multimedia point of view.
