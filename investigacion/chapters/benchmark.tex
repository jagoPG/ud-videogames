\chapter{Benchmark}
As long as \emph{Vulkan} approaches to be an \emph{OpenGL} replacement, it is required to compare if \emph{Vulkan} is
able to offer
a throughput greater than \emph{OpenGL} - or at least similar. For this reason I have performed a benchmark with two
different games. With each game I have played twice a round, one time with \emph{OpenGL} and another time with
\emph{Vulkan}. I have played the same scenes with the two implementations of the technologies for avoid having less
computing load with one of the technologies. The game screen resolution has been set up \emph{1920x1080}, with the
highest graphic quality.

The specifications of the computer I have done the tests with are the following: \emph{Intel Core i7-4790} processor,
16 GB of DDR3 RAM running at 1,600 MHz. and an \emph{AMD Radeon R9 380 Series} graphic card. I have updated the
\emph{AMD} drivers to the latest version for having the latest improvements.

I have used the software \emph{PresentMon64} for getting the measurements, which is free for use in the \emph{GitHub}
repository \href{https://github.com/GameTechDev/PresentMon}{GameTechDev/PresentMon} at \emph{GitHub}. This application
can measure the times that takes to draw the frame, time that takes to render the frame, synchronization time...

\begin{table*}[t]
  \centering
  \begin{tabular}{p{0.15\linewidth}p{0.15\linewidth}p{0.15\linewidth}p{0.15\linewidth}p{0.15\linewidth}p{0.15\linewidth}}
    \toprule
    Juego & \gls{api}    & Average & Maximum & Minimum & Standard Deviation \\
    \midrule
    Doom 2016   & OpenGL  & 149 & 184 & 7   & 11.169 \\
    Doom 2016   & Vulkan  & 85  & 194 & 36  & 9.786 \\
    Dota 2      & OpenGL  & 137 & 200 & 6   & 33.17 \\
    Dota 2      & Vulkan  & 122 & 200 & 9   & 30.719 \\
    \bottomrule
  \end{tabular}
  \caption{Benchmark Results. Measured as \gls{fps}.}
  \label{tab:benchmark_results}
\end{table*}

During the test I have not notice any problem when playing - that is the game experience has been excellent. Based
on the collected data the Table \ref{tab:benchmark_results} has been constructed. As you can see, the average of
OpenGL has been greater that \emph{Vulkan} average. However, \emph{Vulkan} has a greater frame stability as the
standard deviation is lower than the \emph{OpenGL} frame rate. Both technologies have a greater frame average than the
frame rate the players currently demand - about 60 \gls{fps} playing with a resolution of \emph{1920x1080}.

It has to be taken into consideration that \emph{Vulkan} drivers have not the same maturity level as \emph{OpenGL}. In
the future it is for sure that the \emph{Vulkan} results will be improved. As conclusion of this test, I think that
today \emph{Vulkan} could be a good alternative to \emph{OpenGL}, at least from the multimedia point of view.
