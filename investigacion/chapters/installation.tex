\chapter{A quick look}
Si este artículo le ha llamado la atención, voy a proponer una manera de echarle un vistazo más de cerca al kit de
desarrollo de Vulkan. Para ello propongo el repositorio de GitHub \href{https://github.com/LunarG/VulkanSamples}{LunarG/VulkanSamples}
donde se pueden encontrar varios ejemplos ejecutables con este motor.

En primer lugar hay que preparar las herramientas necesarias para poder desarrollar. En primer lugar hay que asegurarse
que la tarjeta gráfica soporta Vulkan, en la tabla \ref{tab:vulkan_drivers} se puede comprobar si su modelo de tarjeta
gráfica tiene drivers que den soporte a Vulkan.

La preparación del entorno de desarrollo es diferente en Windows y GNU/Linux, pero en general los ingredientes a
tener listos son:
\begin{enumerate}
    \item CMake: compilador de C/C++.
    \item Python 3.
    \item Git
    \item Glslang: necesario para compilar GLSL a SPIRV.
\end{enumerate}

En Windows faltaría instalar un \gls{ide}, que en este entorno podría ser Visual Studio por excelencia, y descargar
el \gls{sdk} de Vulkan de la página web de \emph{LunarG}. Al crear un proyecto nuevo en C++ es necesario indicar la ruta
donde está situado el kit de desarrollo, y añadir a las dependencias las librerías
\emph{vulkan-1.lib} y \emph{glfw3.lib}.

Por otra parte en GNU\/Linux, hay que descargarse el \gls{sdk} de Vulkan de la página web de \emph{LunarG} e instalarlo
mediante el ejecutable que trae en el paquete. Es necesario instalar una serie de librerías además de las dependencias
indicadas anteriormente:

\begin{enumerate}
    \item libxcb1-dev: librería XCB, se emplea para comunicarse con el sistema \emph{X-Window}.
    \item xorg-dev: ficheros de desarrollo de Xorg.
    \item libglm-dev: librería GLM, se emplea para realizar operaciones de algebra lineal.
\end{enumerate}

Una vez realizadas las instalaciones, hay que proceder a clonar el repositorio donde se situa el ejemplo. Cuando se
haya descargado el proyecto, hay que ejecutar el fichero \emph{update\_external\_sources} el cual descarga y compila
las dependencias \emph{glslang} y \emph{spriv-tools} que requiere este proyecto. Finalmente solo hay que compilar
y ejecutar el proyecto, tal y como se indica en el fichero \emph{README.md} que se puede encontrar en el proyecto.

\begin{table*}[t]
  \centering
  \begin{tabular}{p{0.25\linewidth}p{0.25\linewidth}p{0.25\linewidth}}
    \toprule
    Tarjeta gráfica & Driver Windows    & Driver GNU/Linux \\
    \midrule
    NVIDIA          & 368.69+                   & 367.27+ \\
    AMD             & Crimson Edition 16.3+     & Crimson Edition 16.3+ \\
    Intel           & 15.40.20.4404+            & Mesa 12.0+ \\
    \bottomrule
  \end{tabular}
  \caption{Drivers de Vulkan}
  \label{tab:vulkan_drivers}
\end{table*}
