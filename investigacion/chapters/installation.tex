\chapter{A quick look}
If this article has attracted your attention, I am going to propose you a way to take a closer look to the development
kit of Vulkan. \href{https://github.com/LunarG/VulkanSamples}{LunarG/VulkanSamples} GitHub repository contains several
executable examples using the LunarG implementation of Vulkan.

In the first place it is required to prepare the tools for developing. Ensure that the graphic card supports Vulkan. In
the table \ref{tab:vulkan_drivers} can be checked if your model of graphic card has drivers that provide support to
Vulkan. Very few integrated cards of Intel cards support it, so make sure you are using a good graphic card. So if you
find problems when compiling the source, the problem can be cause for a lack of support.

Preparation of the development environment is different in Windows and GNU\/Linux, but generally the ingredients are:
\begin{enumerate}
    \item CMake: C/C++ compiler.
    \item Python 3.
    \item Git
    \item Glslang: required for compiling compilar \gls{GLSL} to \gls{SPIRV}.
\end{enumerate}

Under Windows you have to install an \gls{ide}, which in this environment could be Visual Studio par excellence. Then,
download and install the LunarG \gls{sdk} Vulkan implementation. After creating a new C++ project it is required to
point out where the \gls{sdk} is, and add the library dependencies \emph{vulkan-1.lib} and \emph{glfw3.lib}.

On the GNU\/Linux hand, download and install the LunarG \gls{sdk} Vulkan implementation. There is an executable file
in the package that eases the installation tasks. It is required to install some libraries in addition to the
previous dependencies:
\begin{enumerate}
    \item libxcb1-dev: XCB library, employed for communicating with the \emph{X-Window} system.
    \item xorg-dev: Xorg development files.
    \item libglm-dev: GLM library, employed for lineal algebra operations.
    \item mesa-dev: MESA development libraries.
\end{enumerate}

After all requirements are install, proceed to clone the repository where the example is located. After the project
is downloaded, the \emph{update\_external\_sources} file has to be executed. This script downloads and compiles
all dependencies that require this project (\emph{glslang} and \emph{spriv-tools}). Finally, the project has to be
compiled and executed, such as it is indicated in the \emph{README.md} file which can be found in the project.

\begin{table*}[t]
  \centering
  \begin{tabular}{p{0.25\linewidth}p{0.25\linewidth}p{0.25\linewidth}}
    \toprule
    Tarjeta gráfica & Windows    & GNU\/Linux \\
    \midrule
    NVIDIA          & 368.69+                   & 367.27+ \\
    AMD             & Crimson Edition 16.3+     & Crimson Edition 16.3+ \\
    Intel           & 15.40.20.4404+            & Mesa 12.0+ \\
    \bottomrule
  \end{tabular}
  \caption{Vulkan Drivers}
  \label{tab:vulkan_drivers}
\end{table*}
